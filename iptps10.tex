\documentclass[letterpaper,twocolumn,10pt]{article}
\usepackage{times}
\usepackage[small,compact]{titlesec}
\usepackage[small,it]{caption}
\usepackage{usenix}
\usepackage{endnotes}
\usepackage[tight]{subfigure}
\usepackage{natbib}
\setlength{\bibsep}{0.0pt}

\usepackage{url}
\usepackage{multirow}
\usepackage{array}
\usepackage{epsfig}
\usepackage{footnote}
\usepackage{amsmath}

\usepackage[compact]{titlesec}
\titlespacing{\section}{0pt}{*.5}{*.5}
\titlespacing{\subsection}{0pt}{*.5}{*.5}
\titlespacing{\subsubsection}{0pt}{*.5}{*.5}
\setlength{\parskip}{0pt}
\setlength{\parsep}{0pt}
\setlength{\headsep}{0pt}
\setlength{\topskip}{0pt}
\setlength{\topmargin}{0pt}
\setlength{\topsep}{0pt}
\setlength{\partopsep}{0pt}
%\widowpenalty=10000
%\clubpenalty=10000
%\setlength{\parskip}{0pt}
%\setlength{\dbltextfloatsep}{.2cm}
%\setlength{\dblfloatsep}{.2cm}
%\setlength{\textfloatsep}{.2cm}
%\setlength{\floatsep}{.2cm}
%\setlength{\topsep}{.2cm}
%\setlength{\intextsep}{.2cm}
%\setlength{\belowcaptionskip}{.2cm}
%\renewcommand{\topfraction}{0.85}
%\renewcommand{\textfraction}{0.1}
%\renewcommand{\floatpagefraction}{0.85}
%\hyphenpenalty=5000
%\tolerance=1000

\begin{document}

\title{\Large \bf Towards Social Profile Based Overlays}

\author{
David Isaac Wolinsky,
Pierre St. Juste,
P. Oscar Boykin,
Renato Figueiredo
\\
University of Florida
\\
}

%\maketitle


\twocolumn[%
\centerline{\Large \bf Towards Social Profile Based Overlays}

\medskip

\centerline{\bf 
  David Isaac Wolinsky,
  Pierre St. Juste,
  P. Oscar Boykin,
  Renato Figueiredo
}
\centerline{
  University of Florida
}
\bigskip
]

\subsection*{Abstract}
% Use Online Social Networks instead of social networks
Social networking has quickly become one of the most common activities of
Internet users. As social networks evolve, they request more information from
the users and thus requiring the users to place more trust into the social
network. Peer-to-peer (P2P) overlays can return ownership of information and
system control to the user as they can be constructed in a way to not require
third party proxies.

In this paper, we present a novel concept known as the structured social overlay
that applies social networks to structured P2P overlays to provide ownership,
scalability, reliability, and security. Each user's profile is assigned a
unique private, secure overlay, where members of that overlay have a friendship
with the overlay owner. The profile data is stored using the profile overlay's
distributed data stores. To ensure privacy, the profile overlay employs the
public key infrastructure, where the role of certificate authority (CA) is
handled by the overlay owner and each member of the overlay has a CA signed
certificate. Each member of the social network, joins a common public overlay,
which provides services to discover friends and bootstrap connections into
existing private overlays through a distributed data store. We define interfaces
that can be used to implement this system as well as explore some of the
challenges related to it.

\section{Introduction}
Social networking has become pervasive in daily life, though as social networks
grow so does the wealth of personal information that they store.  Users become
more dependent on social networks as users surrender tasks such as communication
and identity to the social network.  Once information has been released to a
social network, known as a user's profile, the user is at the mercy of the
social network.  If the social network engages in activites disagreeable to the
user, such as recent issues with Facebook's Beacon program~\cite{facebook_beacon},
the user has the option to leave the social network surrendering their identity
and features provided by the social network, to accept the disagreeable
activities, or to petition and hope that the social network changes its behavior.

Recent work~\cite{p2p_socialnetwork} presents the coupling between social
networks and P2P systems.  Noting that a social network made up of social
links is inherently a P2P system with the aside that they are currently
developed on top of centralized systems.  In this paper, we focus on the
topic of topology, how the P2P system should be organized, by applying the
social network to structured P2P overlays.  We then explain how the use of
this topology enables reliability, security, and robustness.

Structured P2P overlays provide a scalable, resilient, and self-managing
platform for distributed applications.  Structured overlays provide means
by which users can easily create their own decentralized systems for the
purpose of data sharing, interactive activities, and other networking enabled
activities.  In recent work~\cite{icdcs10}, we have implemented mechanisms that
allow users to create and manage their own private overlays using a common
public overlay to assist in discovery and NAT traversing of the private overlay.
In this paper, we further this work by an indepth discussion on how to apply
this technique to social networks.

Social networks consist of users, each of whom typically has a a profile, a set
of friends, and a private message inbox.  The profile contains the users
personal information, status updates, and public conversation with the friends.
Friends are individuals which the user trust sufficiently to view the profile.
The private message inbox allows users to send messages between each other
without leaking any information to other friends.  Using this model, we
describe a common directory or public overlay which allows peers to provide
services where peers can find friends and join overlays where there already
exists an established relationship.  Each user has their own profile overlay,
where the members of the overlay are limited to the current friends of the
profile owner.  The profile overlay is secured by a public key infrastructure
(PKI) with the profile owner being the certificate authority (CA).  The profile
information is stored information is stored in distributed datastores, allowing
profile information to be accessed in scalable mechanisms regardless of the
profile owner's online state.
% Not sure about the above statement

Our proposed social network makes use of two types of overlays, a directory
overlay and a profile overlay.  In this paper, we explain how these overlays
provide key features of social networks:  finding and befriending peers,
sending public and private messages, and sharing media.

The rest of this paper is organized as follows.  Section~\ref{background}
provides background and related work.  Section~\ref{social_overlays} describes
our multioverlay approach, explaining how to map social networks onto structured
P2P overlays.  In Section~\ref{constraints}, we present constraints that our
proposed system introduces and provide potential solutions.  We conclude the
paper in Section~\ref{conclusion}.

\section{Background}
\label{background}
In this section, we review structured P2P overlays and distributed and
decentralized social network techniques.
\subsection{Structured P2P Overlays}
Structured P2P systems provide distributed lookup services with guaranteed
search time with a lower bound of $O(\log N)$, in contrast to unstructured
systems, which rely on global knowledge/broadcasts, or stochastic techniques
such as random walks~\cite{unstructured_v_structured}.  Some examples of
structured systems can be found in~\cite{pastry, chord, symphony, kademlia,
can}.  In general, structured systems are able to make these guarantees by
self-organizing a structured topology, such as a 2D ring or a hypercube.

In the overlay, each node is given a unique node ID drawn from a large address
space.  Each node id must be unique otherwise address collisions will occur,
which can prevent nodes from participating in the overlay.  Furthermore, having
the node IDs well distributed assist in providing better scalability as many
shortcut selection algorithms depend on having node IDs uniformly distributed
across the entire address space.  A simple mechanism to ensure this behavior is
to have each node use a cryptographically strong random number generator to
generate the node ID.  Another mechanism involves the use of a trusted third
party to generate node IDs and cryptographically sign them~\cite{secure_routing}.

As with all P2P systems, in order for an incoming node to connect with
the overlay, the node must know of at least one active participant.  A list of
nodes that are running on public addresses is typically distributed with the
application, available through some out-of-band mechanism, or possibly using
multicast to findpools~\cite{pastry}.

In dealing with ring based overlays, a node must be connected to closest
neighbors in the node ID address space; optimizations for fault tolerance
suggest that it should be between 2 to $\log(N)$ on both sides.  Having
multiple peers on both sides assist in stabilizing the overlay structure
when experiencing churn, particularly when peers leave without warning.

Overlay shortcuts enable efficient routing in ring-structured P2P systems.
Different shortcut selection methods include: maintaining large tables without
using connections and only verifying usability when routing
messages~\cite{pastry, kademlia}, maintaining a connection with a peer every
set distance in the P2P address space~\cite{chord}, or using locations drawn
from a harmonic distribution in the node address space~\cite{symphony}.

Most structured P2P overlays support decentralized storage/lookup of information by
mapping keys to specific node IDs in an overlay.  At a minimum, the data is stored
at the node ID either smaller or larger to the data's node ID and for fault
tolerance the data can be stored at other nodes.  This sort of mapping
and data storage is called a distributed hash table (DHT).

In \cite{one_ring, randpeer, can_multicast}, the authors
discuss the concept a single overlay supporting services by additional overlays
that use the underlying overlay to assist in discovery.  In \cite{icdcs10}, we
describe a reference implementation of a multiple overlay system that supports
the use of a public overlay's DHT to store currently active peers in the private
overlays.  Whereby users could create their own overlays without having to
create their own bootstrap network.  In addition, our system provides both relay
and hole-punching NAT traversal techniques and supports point-to-point PKI
based security.

\subsection{Peer-to-Peer Social Networks}
The recent popularity and growing privacy issues of centralized online social networks 
has motivated research projects aimed at providing private, peer-to-peer
social networks~\cite{peerson, matryoshkas, tribler-osn, vis-a-vis}.
In~\cite{peerson}, a DHT provides the lookup service for storing metadata 
pertaining to a peer's profile. Peers query the DHT for updated content from 
their friends by hashing their unique identifiers (e.g. friends' email addresses). 
The retreived metadata contains information for obtaining the profile data such as
IP address, file version, and so on. There is also assumption of a PKI system which
is their basis for identification, encryption, and access control. In our
approach, a private overlay is created for each peer and access control is
based on admission to the private overlay. Once admitted to the private
overlay of a particular peer, users can acess the overlay-specific DHT for 
peer resources. Also all communication between peers are encrypted at the
overlay level based on PKI-based certificate exchange amongst peers.

\cite{vis-a-vis} takes a different approach by depending on virtual individual
servers (VIS) hosted on a cloud infrastructure such as Amazon EC2. Friends contact
each other's VIS directly for updates and uses a DHT only as a directory for
groups, or interest-based searches. Their approach assumes bidirectional
end-to-end connectivity between each VIS, which is a reasonable assumption if VIS 
is hosted on a cloud infrastructure. If users decided to host their VIS on their personal
machines, connectivity would become an issue. Our approach uses a public overlay
which requires a subset of the nodes to be publicly Internet addressable. The
public overlay then enables the construction of a private overlay by facilitating
NAT traversal or relaying to ensure all-to-all connectivity between all nodes.

The matryoshka approach presented in~\cite{matryoshka} also uses a DHT for
looking up a peer's matryoshka (or circle of friends). Once a node in the 
peer's outermost circle is found, that node is used to route profile requests to 
the innermost circle which contains replicas of a peer's profile. This
approach depends on the availability of a trusted identification service
contacted through the DHT. This approach focuses on anonymity, access control,
and availability. \cite{matryoshka}'s circle of friends concept
is somewhat related to our idea of a private overlay. But in our overlay,
the peer serves as the CA of its overlay, and a node can only join the
overlay after the peer has provided it with a signed-certificate which binds
the P2P address to the peer. Also, they do not deal with connectivity issues
between nodes.

All of the previous works relied on DHTs built on top of structured 
peer-to-peer overlays. The peer-to-peer social network presented
in~\cite{tribler-osn} is based on an unstructured social overlay
and does not require a DHT. Peers connect to each other directly over IP
without any overlay routing. Once peers are friends, they maintain
unique identifiers of each to deal with dynamic IPs. Peers cache each
other's data for availability through replication, and helper nodes
are used to assist with communication between NATted peers. This
approach did not talk about security or access control. We deal with
both security and access control at the overlay layer with PKI based
concepts. Overall, no previous work has looked into creating profile-specific
overlays for users with the owner of the profile serving as the CA
for its overlay. By limiting access and providing encryption at the overlay
layer, users can leverage many existing structured overlay solutions without
dealing with typical trusts issues of associated with peer-to-peer networks.

\section{Social Overlays}
\label{social_overlays}
In this section we introduce the components of our multioverlay system,
the public directory overlay and the private profile overlays.  A directory
overlay supports two features:  1) a directory for friend discovery and
verification and 2) lists of peers currently active in each profile overlay.
A profile overlay supports the features of a profile, private messages, and
media sharing.  In this section, we explain how to map social networking
features to this multioverlay approach.  First we explain how peers find
each other, then there interaction in the private overlay, and finally how
they connect to the private overlay.

\subsection{Finding and Verifying Friends}
In a traditional social network, a directory consists of many directory entries
consisting of peer's public information, such as the user's name, user name,
e-mail address, group affiliations, and friends.  A directory can be searched
using this information to find one or more matching directory entries.  The user
then makes a friendship request, which notifies the remote peer of this request.
The request receiver can review the public information of the requestor prior to
making a decision.  If the receiver accepts the request, the peers are given
access to each other's profiles.  Whereupon, they can learn more information
and if it turns out to be a mistake, the peers can unilaterally end the
relationship.

To map this to our proposed social overlay, the directory entries can be
inserted into the DHT.  As discussed in previous work, the keys where the
directory entries are stored at consist of a subset of the user's public
information in lower-case format and hashed to an overlay  address.  The value
stored at these keys is the user's certificate, which consists of its public
information and an overlay address where the user expects to receive
notifications.  The overlay address can be used for asynchronous offline
messaging, whose function we will explain shortly.

To bind public information to a certificate, we use a certificate of the format
presented in Figure~\ref{fig:certificate}.  The main portion of the certificate is 
similar to a self-signed x509~\cite{x509} certificate with public information
such as user's name, user name, e-mail address, and group affiliations embeded
into the certificate.  A friend list is represented by many friend entries, for
this we employ a technique similar to PGP, user's can acquire from their friends
a signed message consisting of a hash of the peers certificate, the time stamp,
and the friend's certificate hash signed by the friend.  Since PGP does not
provide a strong method for revocation the time stamp provides a slightly better
means to decipher whether or not a friendship link is still active without
accessing the public overlay of either peers.  Peers should request new friend
list entry within a certain period of time or it will appear that the friendship
is no longer valid.

While looking for an individual, a peer may discover that many individuals have
overlapping public information components, such as the user's name.  Assuming
all entries are legitimate, the overlay must support inserting multiple values
at the same key, leaving the peer or the peer's DHT client to parse the
responses and determining the best match by reviewing the contents of each
certificate.  Alternatively a technique like Sword~\cite{sword} support
distributing the data across a set of nodes and using a bounded broadcast to
discover peers that match all information used for searching.

Upon discovering an individual with whom a peer would like a friendship, the
peer will issue a friendship request.  As stated earlier, the data stored in
the directory has an overlay address, where a peer expects friendship requests
to be inserted into the DHT.  The friendship request consists of the self-signed
certificate of the requesting peer, the public information of the request
receiver, a time stamp, and a signature made from the private key associated with
the self-signed certificate.  Though because DHTs are soft state systems having
leases, the requester must reinsert the request upon timeout and no response for
the receiver.

Once a request has been inserted into the DHT, the receiver can come online and
check for outstanding requests.  If the receiver would like to add the user, he
may do so conditionally or unconditially.  An unconditional accept would cause
the user to issue a request himself and also sign the request of the originating
requester.  Alternatively in the case of a conditional accept, the user would
issue a request, wait for a reply, and investigate prior to signing the originating
requesters request.  Once a user has received signed certificate, they may access
the remote peers profile overlay as discussed in~\ref{profile_overlay}, which is
also responsible for activities such as revocation.

Since a DHT is a soft-state system that uses leases to remove expired data, requests
and responses must be occassionally reinserted into the DHT.  Alternatively
approaches mentioned in~\cite{}, that suggest methods of storing data in overlays
using quotas could be used to ensure fair usage of the overlay.

\subsection{The Profile Overlay}
\label{profile_overlay}
In a traditional social network, the profile or user-centric portion consists
of a public message board for status updates or public messages, private
messaging, data sharing, and maintenance of existing friendships.  In this
section, we explain how these components can be applied to a structured overlay
dedicated to an individual profile.

The profile overlay consists of all the online friends of the profiler owner.
For access control, we employ a PKI, where each member uses the signed certificate
generated during the ``finding and verifying friends'' stage.  All links are
encrypted using symmetric security algorithms established through the PKI.
Thus preventing uninvited guests from gaining direct access to the overlay.
Because the profile owner also is the CA for all members of the overlay, they
can easily revoke users from the profile overlay with ease similar to a
centralized social network.  In ~\cite{icdcs}, we presented a broadcast mechanism
for immediately revoking peers from a system and we also suggest using a DHT based
revocation list for indirect revocation.

The message board of a profile can be stored in two ways, distributed via a
data store as well as stored on the profile owners personal computing devices.
While the distributed data store would provide the profile when the profile
owner is offline, it also distributes the load for popular profiles.  Since
reliability in a profile may be suspect, each peer should always be a provider
for all data in their profile, when they are online.

Similarly, private messages could be stored in this distributed data store,
though since they are private, the message should be prepended by symmetric
key used to encrypt the rest of the message.  the symmetric key should be
encrypted with the profile owner's public key, so that they are the only party
that can read the message.  Additionally, the body of the message encrypted
should contain all relevant material such as the sender, time sent, subject,
and the message itself.  If the messages are stored in a well known location,
the profile owner can either poll the location or, alternatively, use an event
based system to notify them of new messages.

An ideal distributed data store would support database like access allowing
to be stored chronologically and by type, i.e., media, status update, or public
messages.  The distributed data store should be built on top of the profile
overlay so that only members of the profile store the personal
data of the user.  Since the material is public for the group and the groups
links are all encrypted, the data could be distributed without encryption
though all data should be signed so that each post has an identity attached to
it.  Posts that lack this identity should be ignored when viewing the profile.
Only the profile owner and owners of posts made on the message board should
have the ability to delete the material.

\subsection{Active Peers}
The directory overlay should be used to assist in finding currently active peers
in the profile overlays.  By placing their node IDs at a well known though unique
per profile overlay key in the DHT, active peers can bootstrap incoming peers
into the profile overlay.  We implemented and evaluated this concept
in~\cite{icdcs10}.  Because the profile overlay members all use PKI to ensure
membership, even if malicious peers insert their ID into the active If malicious
peers were to insert their ID into this list, it would be useless as the peer
would only form connections with peers who also have a signed certificate for
that profile.

\section{Conclusion}
\label{conclusion}

\bibliographystyle{abbrv}
\small{
\bibliography{iptps10}
\suppressfloats
}

\end{document}
