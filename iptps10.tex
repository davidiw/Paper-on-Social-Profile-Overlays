\documentclass[letterpaper,twocolumn,10pt]{article}
\usepackage{times}
\usepackage[small,compact]{titlesec}
\usepackage[small,it]{caption}
\usepackage{usenix}
\usepackage{endnotes}
\usepackage[tight]{subfigure}
\usepackage{natbib}
\setlength{\bibsep}{0.0pt}

\usepackage{url}
\usepackage{multirow}
\usepackage{array}
\usepackage{epsfig}
\usepackage{footnote}
\usepackage{amsmath}

\usepackage[compact]{titlesec}
\titlespacing{\section}{0pt}{*.5}{*.5}
\titlespacing{\subsection}{0pt}{*.5}{*.5}
\titlespacing{\subsubsection}{0pt}{*.5}{*.5}
\setlength{\parskip}{0pt}
\setlength{\parsep}{0pt}
\setlength{\headsep}{0pt}
\setlength{\topskip}{0pt}
\setlength{\topmargin}{0pt}
\setlength{\topsep}{0pt}
\setlength{\partopsep}{0pt}
\widowpenalty=10000
\clubpenalty=10000
%\setlength{\parskip}{0pt}
%\setlength{\dbltextfloatsep}{.2cm}
%\setlength{\dblfloatsep}{.2cm}
%\setlength{\textfloatsep}{.2cm}
%\setlength{\floatsep}{.2cm}
%\setlength{\topsep}{.2cm}
%\setlength{\intextsep}{.2cm}
%\setlength{\belowcaptionskip}{.2cm}
%\renewcommand{\topfraction}{0.85}
%\renewcommand{\textfraction}{0.1}
%\renewcommand{\floatpagefraction}{0.85}
%\hyphenpenalty=5000
%\tolerance=1000

\begin{document}

\title{\Large \bf Towards Social Profile Based Overlays}

\author{
David Isaac Wolinsky,
Pierre St. Juste,
P. Oscar Boykin,
Renato Figueiredo
\\
University of Florida
\\
}

%\maketitle


\twocolumn[%
\centerline{\Large \bf Towards Social Profile Based Overlays}

\medskip

\centerline{\bf 
  David Isaac Wolinsky,
  Pierre St. Juste,
  P. Oscar Boykin,
  Renato Figueiredo
}
\centerline{
  University of Florida
}
\bigskip
]

\subsection*{Abstract}
Social networking has quickly become one of the most common activities of
Internet users. As social networks evolve, they request more information from
the users and thus requiring the users to place more trust into the social
network. Peer-to-peer (P2P) overlays can return ownership of information and
system control to the user as they can be constructed in a way to not require
third party proxies.

In this paper, we present a novel concept known as the structured social overlay
that applies social networks to structured P2P overlays to provide ownership,
scalability, reliability, and security. Each user's profile is assigned a
unique private, secure overlay, where members of that overlay have a friendship
with the overlay owner. The profile data is stored using the profile overlay's
distributed data stores. To ensure privacy, the profile overlay employs the
public key infrastructure, where the role of certificate authority (CA) is
handled by the overlay owner and each member of the overlay has a CA signed
certificate. Each member of the social network, joins a common public overlay,
which provides services to discover friends and bootstrap connections into
existing private overlays through a distributed data store. We define interfaces
that can be used to implement this system as well as explore some of the
challenges related to it.

\section{Introduction}
Social networking has become pervasive in daily life, though as social networks
grow so does the wealth of personal information that they store.  Users become
more dependent on social networks as users surrender tasks such as communication
and identity to the social network.  Once information has been released to a
social network, known as a user's profile, the user is at the mercy of the
social network.  If the social network engages in activites disagreeable to the
user, the user has the option to leave the social network surrendering their
identity and features provided by the social network or to accept the
disagreeable activities.

Structured P2P overlays provide a scalable, resilient, and self-managing
platform for distributed applications.  Structured overlays provide means
by which users can easily create their own decentralized systems for the
purpose of data sharing, interactive activities, and other networking enabled
activities.  In recent work~\ref{icdcs10}, we have implemented mechanisms that
allow users to create and manage their own private overlays using a common
public overlay to assist in discovery and NAT traversing of the private overlay.
In this paper, we further this work by an indepth discussion on how to apply
this technique to social networks.

Social networks consist of users, each of whom typically has a a profile, a set
of friends, and a private message inbox.  The profile contains the users
personal information, status updates, and public conversation with the friends.
Friends are individual which the user trust sufficiently to view the profile.
The private message inbox allows users to send messages between each other
without leaking any information to other friends.  Using this model, we
describe a common directory or public overlay which allows peers to provide
services where peers can find friends and join overlays where there already
exists an established relationship.  Each user has their own profile overlay,
where the members of the overlay are limited to the current friends of the
profile owner.  The profile overlay is secured by a public key infrastructure
(PKI) with the profile owner being the certificate authority (CA).  The profile
information is stored information is stored in distributed datastores, allowing
profile information to be accessed in scalable mechanisms regardless of the
profile owner's online state.

The rest of this paper is organized as follows.  Section~\ref{background}
provides background and related work.  Section~\ref{profile_overlay} and
Section~\ref{directory_overlay} describe our contributions, namely, a
private, profile overlay and a public, directory overlay.  We conclude the
paper in Section~\ref{conclusion}.

\section{Background}
\label{background}
In this section, we review structured P2P overlays and distributed and
decentralized social network techniques.
\subsection{Structured P2P Overlays}
Structured P2P systems provide distributed lookup services with guaranteed
search time with a lower bound of $O(\log N)$, in contrast to unstructured
systems, which rely on global knowledge/broadcasts, or stochastic techniques
such as random walks~\cite{unstructured_v_structured}.  Some examples of
structured systems can be found in~\cite{pastry, chord, symphony, kademlia,
can}.  In general, structured systems are able to make these guarantees by
self-organizing a structured topology, such as a 2D ring or a hypercube.

In the overlay, each node is given a unique node ID drawn from a large address
space.  Each node id must be unique otherwise address collisions will occur,
which can prevent nodes from participating in the overlay.  Furthermore, having
the node IDs well distributed assist in providing better scalability as many
shortcut selection algorithms depend on having node IDs uniformly distributed
across the entire address space.  A simple mechanism to ensure this behavior is
to have each node use a cryptographically strong random number generator to
generate the node ID.  Another mechanism involves the use of a trusted third
party to generate node IDs and cryptographically sign them~\cite{secure_routing}.

As with all P2P systems, in order for an incoming node to connect with
the overlay, the node must know of at least one active participant.  A list of
nodes that are running on public addresses is typically distributed with the
application, available through some out-of-band mechanism, or possibly using
multicast to findpools~\cite{pastry}.

In dealing with ring based overlays, a node must be connected to closest
neighbors in the node ID address space; optimizations for fault tolerance
suggest that it should be between 2 to $\log(N)$ on both sides.  Having
multiple peers on both sides assist in stabilizing the overlay structure
when experiencing churn, particularly when peers leave without warning.

Overlay shortcuts enable efficient routing in ring-structured P2P systems.
Different shortcut selection methods include: maintaining large tables without
using connections and only verifying usability when routing
messages~\cite{pastry, kademlia}, maintaining a connection with a peer every
set distance in the P2P address space~\cite{chord}, or using locations drawn
from a harmonic distribution in the node address space~\cite{symphony}.

Most structured P2P overlays support decentralized storage/lookup of information by
mapping keys to specific node IDs in an overlay.  At a minimum, the data is stored
at the node ID either smaller or larger to the data's node ID and for fault
tolerance the data can be stored at other nodes.  This sort of mapping
and data storage is called a distributed hash table (DHT).

In \cite{one_ring, randpeer, can_multicast}, the authors
discuss the concept a single overlay supporting services by additional overlays
that use the underlying overlay to assist in discovery.  In \cite{icdcs10}, we
describe a reference implementation of a multiple overlay system that supports
the use of a public overlay's DHT to store currently active peers in the private
overlays.  Whereby users could create their own overlays without having to
create their own bootstrap network.  In addition, our system provides both relay
and hole-punching NAT traversal techniques and supports point-to-point PKI
based security.

\subsection{Social Networks}

\section{The Directory Overlay}
\label{directory_overlay}
The directory overlay supports two features:  1) a directory for friend
discovery and verification and 2) lists of peers currently active in each
profile overlay.

\subsection{Finding and Verifying Friends}
\subsection{Active Peers}

\section{The Profile Overlay}
\label{profile_overlay}
Each profile overlay resembles a private overlay as discussed in~\cite{icdcs10}.
In this section, we focus on the tasks of controlling overlay membership, i.e.,
handling friendships; distribution and retrieval of profile information; and
private messaging.

\subsection{Handling Friendships}
\subsection{Storing and Retrieving Profile Data}
\subsection{Private Messaging}

\section{Conclusion}
\label{conclusion}

\bibliographystyle{abbrv}
\small{
\bibliography{iptps10}
\suppressfloats
}

\end{document}
