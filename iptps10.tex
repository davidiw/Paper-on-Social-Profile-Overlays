\documentclass[letterpaper,twocolumn,10pt]{article}
\usepackage{times}
\usepackage[small,compact]{titlesec}
\usepackage[small,it]{caption}
\usepackage{usenix}
\usepackage{endnotes}
\usepackage[tight]{subfigure}
\usepackage{natbib}
\setlength{\bibsep}{0.0pt}

\usepackage{url}
\usepackage{multirow}
\usepackage{array}
\usepackage{epsfig}
\usepackage{footnote}
\usepackage{amsmath}

\usepackage[compact]{titlesec}
\titlespacing{\section}{0pt}{*.5}{*.5}
\titlespacing{\subsection}{0pt}{*.5}{*.5}
\titlespacing{\subsubsection}{0pt}{*.5}{*.5}
\setlength{\parskip}{0pt}
\setlength{\parsep}{0pt}
\setlength{\headsep}{0pt}
\setlength{\topskip}{0pt}
\setlength{\topmargin}{0pt}
\setlength{\topsep}{0pt}
\setlength{\partopsep}{0pt}
%\widowpenalty=10000
%\clubpenalty=10000
%\setlength{\parskip}{0pt}
%\setlength{\dbltextfloatsep}{.2cm}
%\setlength{\dblfloatsep}{.2cm}
%\setlength{\textfloatsep}{.2cm}
%\setlength{\floatsep}{.2cm}
%\setlength{\topsep}{.2cm}
%\setlength{\intextsep}{.2cm}
%\setlength{\belowcaptionskip}{.2cm}
%\renewcommand{\topfraction}{0.85}
%\renewcommand{\textfraction}{0.1}
%\renewcommand{\floatpagefraction}{0.85}
%\hyphenpenalty=5000
%\tolerance=1000

\begin{document}

\title{\Large \bf Towards Social Profile Based Overlays}

\author{
David Isaac Wolinsky,
Pierre St. Juste,
P. Oscar Boykin,
Renato Figueiredo
\\
University of Florida
\\
}

%\maketitle


\twocolumn[%
\centerline{\Large \bf Towards Social Profile Based Overlays}

\medskip

\centerline{\bf 
  David Isaac Wolinsky,
  Pierre St. Juste,
  P. Oscar Boykin,
  Renato Figueiredo
}
\centerline{
  University of Florida
}
\bigskip
]

\subsection*{Abstract}
% Use Online Social Networks instead of social networks
Online social networking has quickly become one of the most common activities of Internet users.
As social networks evolve, they encourage users to share more information,
requiring the users, in turn, to place more trust into the social network.
Peer-to-peer (P2P) overlays provide an environment that can return ownership of
information, trust, and control to the users, away from centralized third-party
social networks.

In this paper, we present a novel concept, social profile overlays,
which enable users to share their profile only with trusted peers in a scalable,
reliable, and private manner.  Each user's profile consists
of a unique private, secure overlay, where members of that overlay have a
friendship with the overlay owner. Profile data is made available regardless of
the state of the profile owner through the use of the profile overlay's
distributed data store.  Privacy and security are enforced through the use of a
public key infrastructure (PKI), where the role of certificate authority (CA) is
handled by the overlay owner and each member of the overlay has a CA-signed
certificate.  To discover friends and bootstrap connections into the profile
overlay, each member of the social network joins a common public or directory
overlay.  We define interfaces and present tools that can be used to implement
this system, as well as explore some of the challenges related to it.

\section{Introduction}
Online social networking has become pervasive in daily life, though as social
networks grow so does the wealth of personal information that they store.  Once
information has been released on a social network, known as a user's profile,
the data and the user are at the mercy of the terms dictated by the social network 
infrastructure, which today is typically third-party, centrally owned.  If the social
network engages in activites disagreeable to the user, due to change of terms or opt-out programs not well understood by users such as recent issues
with Facebook's Beacon program~\cite{facebook_beacon}, the options presented to the user are limited:
to leave the social network (surrendering their identity and features provided
by the social network), to accept the disagreeable activities, or to petition
and hope that the social network changes its behavior. 

As the use of social networking expands to become the primary way in which users
communicate and express their identity amongst their peers, the users become
more dependent on the policies of social network infrastructure owners.  Recent
work~\cite{p2p_socialnetwork} explores the coupling between social networks and
P2P systems as a means to return ownership to the users, noting that a social
network made up of social links is inherently a P2P system with the aside that
they are currently developed on top of centralized systems.  In this paper, we
extend this idea with focus on the topic of topology; that is, how to self-organize
social profiles that leverage the benefits offered by a structured P2P overlay abstraction.

Structured P2P overlays provide a scalable, resilient, and self-managing
platform for distributed applications.  Structured overlays enable users to
easily create their own decentralized systems for the purpose of data sharing,
interactive activities, and other networking-enabled activities.  In recent
work~\cite{icdcs10}, we have implemented mechanisms that allow users to create
and manage their own private overlays using a common public overlay to assist
in discovery and NAT traversal. This prior work focuses on generic structured
P2P private overlays; in this paper, we
expand upon this approach with in-depth discussion on how to apply this technique to
enable social network overlay profiles.

Social networks consist of users, each of whom has a profile, a set
of friends, and private messaging.  The profile contains the user's
personal information, status updates, and public conversations.  Friends are
individuals which the user trust sufficiently to view their profile.  Private
messaging allows users to send messages between each other without leaking any
information to other friends.  Using this model, we describe how a common
directory overlay can be used to provide services for finding friends and
joining existing profile overlays.  Each user has their own profile overlay,
secured via public key infrastructure (PKI) to which they are the certificate
authority (CA).  The profile data is stored in the profile overlay's distributed
datastore, allowing profile information to be accessed in scalable mechanisms
regardless of the profile owner's online state.  In this paper, we present
the architecture of these overlays and how they are used to find and befriend
peers, and describe approaches to handling profile data, and establishing
initial connections to profile overlays.

The rest of this paper is organized as follows.  Section~\ref{background}
provides background and related work.  Section~\ref{social_overlays} describes
our multi-overlay approach, explaining how to map social networks onto structured
P2P overlays.  In Section~\ref{outstanding}, we explore some of the remaining
challenges confronted by our system.  We conclude the paper in
Section~\ref{conclusion}.

\section{Background}
\label{background}
In this section, we review structured P2P overlays and distributed and
decentralized online social network techniques.
\subsection{Structured P2P Overlays}
Structured P2P systems provide distributed lookup services with guaranteed
search time with a lower bound of $O(\log N)$, in contrast to unstructured
systems, which rely on global knowledge/broadcasts, or stochastic techniques
such as random walks~\cite{unstructured_v_structured}.  Some examples of
structured systems can be found in~\cite{pastry, chord, symphony, kademlia,
can}.  In general, structured systems are able to make these guarantees by
self-organizing a structured topology, such as a 2D ring or a hypercube.

In the overlay, each node is given a unique node ID drawn from a large address
space.  Each node ID must be unique otherwise address collisions will occur,
which can prevent nodes from participating in the overlay.  Furthermore, having
the node IDs well distributed assists in providing better scalability as many
shortcut selection algorithms depend on having node IDs uniformly distributed
across the entire address space.  Two approaches for to ensure this behavior are
to have each node use a cryptographically strong random number generator to
generate the node ID.  or to use a trusted third party generate node IDs and
cryptographically sign them~\cite{secure_routing}.

As with all P2P systems, in order for an incoming node to connect with
the overlay, the node must know of at least one active participant.  A list of
nodes that are running on public addresses is typically distributed with the
application, available through some out-of-band mechanism, or possibly using
multicast to findpools~\cite{pastry}.

In dealing with ring based overlays, a node must be connected to closest
neighbors in the node ID address space; optimizations for fault tolerance
suggest that it should be between 2 to $\log(N)$ on both sides.  Having
multiple peers on both sides assists in stabilizing the overlay structure
when experiencing churn, particularly when peers leave without warning.

Overlay shortcuts enable efficient routing in ring-structured P2P systems.
Different shortcut selection methods include: maintaining large tables without
using connections and only verifying usability when routing
messages~\cite{pastry, kademlia}, maintaining a connection with a peer every
set distance in the P2P address space~\cite{chord}, or using locations drawn
from a harmonic distribution in the node address space~\cite{symphony}.

Most structured P2P overlays support decentralized storage/lookup of information by
mapping keys to specific node IDs in an overlay.  At a minimum, the data is stored
at the node ID either smaller or larger to the data's node ID and for fault
tolerance the data can be stored at other nodes.  This sort of mapping
and data storage is called a distributed hash table (DHT).  DHTs provide the
building blocks to form more complex distributed data stores as presented in
Past~\cite{past}.

In \cite{one_ring, randpeer, can_multicast}, the authors discuss the concept a
single overlay supporting services through the use of additional overlays
that use the underlying overlay to assist in discovery.  In \cite{icdcs10}, we
describe a reference implementation of a multiple overlay system that supports
the use of a public overlay's DHT to store currently active peers in the private
overlays.  Whereby users could create their own overlays without having to
create their own bootstrap network.  In addition, our system provides both relay
and hole-punching NAT traversal techniques and supports point-to-point PKI
based security.

\subsection{Peer-to-Peer Social Networks}
The recent popularity and growing privacy issues of centralized online social
networks has motivated research projects aimed at providing private, P2P social
networks~\cite{peerson, matryoshka, tribler-osn, vis-a-vis}.

In~\cite{peerson}, a DHT provides the lookup service for storing metadata
pertaining to a peer's profile. Peers query the DHT for updated content from 
their friends by hashing their unique identifiers (e.g. friends' email
addresses).  The retreived metadata contains information for obtaining the
profile data such as IP address, file version, and so on. Their work relies
on a PKI system that provides identification, encryption, and access control.
In contrast, our approach provides each user their own private overlay secured
by point-to-point encryption and authentication amongst all peers in the profile
overlay.  The profile overlay provides a clean abstraction of access control,
whereby once admitted to a private overlay, users can access a distributed data
store which holds the contents of the owners profile.

\cite{vis-a-vis} takes a different approach by depending on virtual individual
servers (VIS) hosted on a cloud infrastructure such as Amazon EC2. Friends
contact each other's VIS directly for updates and uses a DHT as a directory for
groups or interest-based searches. Their approach assumes bidirectional
end-to-end connectivity between each VIS, where a profile is only available
during the uptime of the VIS.  Because of the demands on network connectivity
and uptime, the approach has difficulty being used on user owned resources.
Our approach enables users to avoid the need for all-to-all connectivity and
constant uptime through the use of extensive NAT traversal support and the
ability to store the profile in the overlay's distributed data store.

The matryoshka approach presented in~\cite{matryoshka} also uses a DHT for
looking up a peer's matryoshka or circle of friends. Once a node in the peer's
outermost circle is found, that node is used to route profile requests to the
innermost circle which contains replicas of a peer's profile. Trust is enabled
through the use of an identification service contacted through the DHT.  The
circle of friends concept lacks the simplicity of the abstraction made in our
approach, whereby any existing overlay without tweaks can be used as a profile
overlay without concern over innermost and outermost circles.  Our approach
also enables the profile owner to serve as a CA whereby nodes can only access
the profile overlay after having obtained a signed certificate.

Unlike the previous works, the P2P social network presented in~\cite{tribler-osn}
is based upon unstructured social overlays and does not require a DHT.  Peers
connect to each other directly over IP without any overlay routing. Once peers
are friends, they maintain unique identifiers to deal with dynamic IPs.  Peers
cache each other's data for availability through replication, and helper nodes
are used to assist with communication between NATted peers.  The approach lacks
security and access control considerations and lacks the guarantees and
simplicity offered by a structured overlay.

\section{Social Overlays}
\label{social_overlays}
In this section we introduce the components of our multioverlay system,
the public directory overlay and the private profile overlays.  A directory
overlay supports two features:  1) a directory for friend discovery and
verification and 2) lists of peers currently active in each profile overlay.
A profile overlay supports the features of a profile, private messages, and
media sharing.  In this section, we explain how to map social networking
features to this multioverlay approach.  First we explain how peers find
each other, then there interaction in the private overlay, and finally how
they connect to the private overlay.

\subsection{Finding and Verifying Friends}
In a traditional social network, a directory consists of many directory entries
consisting of peer's public information, such as the user's name, user name,
e-mail address, group affiliations, and friends.  A directory can be searched
using this information to find one or more matching directory entries.  The user
then makes a friendship request, which notifies the remote peer of this request.
The request receiver can review the public information of the requestor prior to
making a decision.  If the receiver accepts the request, the peers are given
access to each other's profiles.  Whereupon, they can learn more information
and if it turns out to be a mistake, the peers can unilaterally end the
relationship.

To map this to our proposed social overlay, the directory entries can be
inserted into the DHT.  As discussed in previous work, the keys where the
directory entries are stored at consist of a subset of the user's public
information in lower-case format and hashed to an overlay  address.  The value
stored at these keys is the user's certificate, which consists of its public
information and an overlay address where the user expects to receive
notifications.  The overlay address can be used for asynchronous offline
messaging, whose function we will explain shortly.

To bind public information to a certificate, we use a certificate of the format
presented in Figure~\ref{fig:certificate}.  The main portion of the certificate is 
similar to a self-signed x509~\cite{x509} certificate with public information
such as user's name, user name, e-mail address, and group affiliations embeded
into the certificate.  A friend list is represented by many friend entries, for
this we employ a technique similar to PGP, user's can acquire from their friends
a signed message consisting of a hash of the peers certificate, the time stamp,
and the friend's certificate hash signed by the friend.  Since PGP does not
provide a strong method for revocation the time stamp provides a slightly better
means to decipher whether or not a friendship link is still active without
accessing the public overlay of either peers.  Peers should request new friend
list entry within a certain period of time or it will appear that the friendship
is no longer valid.

\begin{figure}[ht]
\centering
\epsfig{file=figs/in_progress.eps, width=2.75in}
\caption{An example certificate for use in our proposed system.}
\label{fig:certificate}
\end{figure}


While looking for an individual, a peer may discover that many individuals have
overlapping public information components, such as the user's name.  Assuming
all entries are legitimate, the overlay must support inserting multiple values
at the same key, leaving the peer or the peer's DHT client to parse the
responses and determining the best match by reviewing the contents of each
certificate.  Alternatively a technique like Sword~\cite{sword} support
distributing the data across a set of nodes and using a bounded broadcast to
discover peers that match all information used for searching.

Upon discovering an individual with whom a peer would like a friendship, the
peer will issue a friendship request.  As stated earlier, the data stored in
the directory has an overlay address, where a peer expects friendship requests
to be inserted into the DHT.  The friendship request consists of the self-signed
certificate of the requesting peer, the public information of the request
receiver, a time stamp, and a signature made from the private key associated with
the self-signed certificate.  Though because DHTs are soft state systems having
leases, the requester must reinsert the request upon timeout and no response for
the receiver.

Once a request has been inserted into the DHT, the receiver can come online and
check for outstanding requests.  If the receiver would like to add the user, he
may do so conditionally or unconditially.  An unconditional accept would cause
the user to issue a request himself and also sign the request of the originating
requester.  Alternatively in the case of a conditional accept, the user would
issue a request, wait for a reply, and investigate prior to signing the originating
requesters request.  Once a user has received signed certificate, they may access
the remote peers profile overlay as discussed in~\ref{profile_overlay}, which is
also responsible for activities such as revocation.

Since a DHT is a soft-state system that uses leases to remove expired data, requests
and responses must be occassionally reinserted into the DHT.  Alternatively
approaches mentioned in~\cite{}, that suggest methods of storing data in overlays
using quotas could be used to ensure fair usage of the overlay.

\subsection{The Profile Overlay}
\label{profile_overlay} In a traditional social network, the profile or
user-centric portion consists of a public message board for status updates or
public messages, private
messaging, data sharing, and maintenance of existing friendships.  In this
section, we explain how these components can be applied to a structured overlay
dedicated to an individual profile.

The profile overlay consists of all the online friends of the profiler owner.
For access control, we employ a PKI, where each member uses the signed certificate
generated during the ``finding and verifying friends'' stage.  All links are
encrypted using symmetric security algorithms established through the PKI.
Thus preventing uninvited guests from gaining direct access to the overlay.
Because the profile owner also is the CA for all members of the overlay, they
can easily revoke users from the profile overlay with ease similar to a
centralized social network.  In ~\cite{icdcs10}, we presented a broadcast mechanism
for immediately revoking peers from a system and we also suggest using a DHT based
revocation list for indirect revocation.

The message board of a profile can be stored in two ways, distributed via a
data store as well as stored on the profile owners personal computing devices.
While the distributed data store would provide the profile when the profile
owner is offline, it also distributes the load for popular profiles.  Since
reliability in a profile may be suspect, each peer should always be a provider
for all data in their profile, when they are online.

Similarly, private messages could be stored in this distributed data store,
though since they are private, the message should be prepended by symmetric
key used to encrypt the rest of the message.  the symmetric key should be
encrypted with the profile owner's public key, so that they are the only party
that can read the message.  Additionally, the body of the message encrypted
should contain all relevant material such as the sender, time sent, subject,
and the message itself.  If the messages are stored in a well known location,
the profile owner can either poll the location or, alternatively, use an event
based system to notify them of new messages.

An ideal distributed data store would support complex queries~\cite{complex_queries}
thus allowing easy access to data stored chronologically, by content, or by type,
i.e., media, status update, or public messages.  The distributed data store
should be built on top of the profile overlay so that only members of the
profile store the personal data of the user.  Since the material is public for
the group and the groups links are all encrypted, the data could be distributed
without encryption though all data should be signed so that each post has an
identity attached to it.  Posts that lack this identity should be ignored when
viewing the profile.  Only the profile owner and owners of posts made on the
message board should have the ability to delete the material.

\subsection{Active Peers}
The directory overlay should be used to assist in finding currently active peers
in the profile overlays.  By placing their node IDs at a well known though unique
per profile overlay key in the DHT, active peers can bootstrap incoming peers
into the profile overlay.  We implemented and evaluated this concept
in~\cite{icdcs10}.  Because the profile overlay members all use PKI to ensure
membership, even if malicious peers insert their ID into the active If malicious
peers were to insert their ID into this list, it would be useless as the peer
would only form connections with peers who also have a signed certificate for
that profile.

\section{Future Work}
\label{outstanding}

\section{Conclusion}
\label{conclusion}
In this paper, we presented methods by which a social network can be
decentralized through the use of structure P2P overlays.  Our system involves
the use of multiple overlays where all users join a common public overlay and
individual profile overlays.  The public overlay provides directory services
that enable users to find and befriend other peers and bootstrap connections
into the secure profile overlays.  Upon forming a friendship through the
directory overlay, peers are given CA signed certificates that allow them to
join each other's profile overlay.  The owner of the profile overlay acts as
CA allowing the user to unilaterally revoke certificate, thus ending
friendships with members of their overlay using efficient and reliable methods.
For the purpose of storing profile information into the overlay, we cite
previous work that can be used to provide distributed data services and give
examples of how to store data securely in the overlay.  Our proposed system
returns control of the social network and more importantly users' identity to
the users and eliminates the need for centralized social networks.

\bibliographystyle{abbrv}
\small{
\bibliography{iptps10}
\suppressfloats
}

\end{document}
